%! Author = joels
%! Date = 24/12/2020

\section{Classes and Operators}
% You can implement your own data types
% You know the elements a class consists of
% You know how to overload operators for classes
% You know the correct way to read and print objects

\subsection{Classes}
\begin{itemize}
    \item A class defines a new type
    \item A class is usually defined in a header file
    \item At the end of a class definition a semicolon is required
    \item Include guard in header file
    \item Keyword: \textcolor{blue}{calss} or \textcolor{blue}{struct}
        \SubItem{Default visibility: Class: private, struct: public}
    \item Access specifiers:
        \SubItem{private: visible only inside the class; for hidden data members}
        \SubItem{protected: also visible nin subclasses}
        \SubItem{public: visible everywhere; for the interface of the class}
    \item Constructor: Initializer list for member initialization
\end{itemize}
\begin{lstlisting}[style=frame, style= linenumbers, language=C]
#ifndef DATE_H_                               #include "Date.h"
#define DATE_H_
class Date {                                  Date::Date(int year, int month, int day)
    int year, month, day;                       : year {year}, month {month}, day {day} {
public:                                         /* ... */
    Date(int year, int month, int day);       }

    static bool isLeapYear(int year);         bool Date::isLeapYear(int year) {
                                                /* ... */
private:                                      }
    bool isValidDate() const;                 bool Date::isValidDate() const {
};                                              /* ... */ }
#endif /* DATE_H_ */                          }
\end{lstlisting}

\subsubsection{Special Constructors}
\begin{itemize}
    \item Default Constructor
        \SubItem{No parameters. Implicitly available if there are no other explicit constructors. Has to initialize member variables with default values.}
    \item Copy Constructor
        \SubItem{Has one $<$own-type$>$ const \& parameter. Implicitly available (unless there is an explicit move constructor or assignment operator). Copies all member variables.}
    \item Move Constructor
        \SubItem{Has one $<$own-type$>$ \&\& parameter. Implicitly available (unless there is an explicit copy constructor or assignment operator). Moves all memebers}
    \item Typeconversation Constructor
        \SubItem{Has one $<$other-type$>$ const \& parameter. Converts the input type if possible. Declare explicit to avoid unexpected conversions.}
    \item Initializer List Constructor
        \SubItem{Has one std::initializer\_list parameter. Does not need to be explicit, implicit conversion is usually desired. Initializer List constructors are preferred if a variable is initialized with $\{$ $\}$}
    \item Destructor
        \SubItem{Named like the default constructor but with a $\sim$. Must release all resources. Implicitly available. Must not throw an exception. Called automatically at the end of the block for local instances.}
\end{itemize}

\begin{lstlisting}[style=frame, style= linenumbers, language=C]
class Date {
public:
    Date(int year, int month, int day);
    Date(); // Default-Constructor
    Date() = default; // explizit Default-Constructor
    Date(Date const &); // Copy-Constructor
    Date(Date &&); // Move-Constructor
    explicit Date(std::string const &); // Typeconversation-Constructor
    Date(std::initializer_list<Element> elements); // Initializer List-Constructor
    ~Date(); // Destructor
    Date(Date const &) = delete; // delete implicit Copy-Constructor
};
\end{lstlisting}

\subsection{Operator Overloading}
\begin{itemize}
    \item Custom operators can be overloaded for user-defined types.
    \item Declared like a function, with a special name: \textcolor{blue}{$<$returntype$>$ operator op($<$parameters$>$);}
    \item Non-Overloadable Operators: \textcolor{red}{::} , \textcolor{red}{.*} , \textcolor{red}{.} , \textcolor{red}{?:}
    \item Keyword \textcolor{blue}{inline} when defined in header file. But: Problem with private varaibles.\\
    $\rightarrow$ Define operator in class
\end{itemize}
\textcolor{blue}{std::tie} creates a tuple and binds the argument with lvalue references. \textcolor{blue}{std::tuple} provides comparison operators: \textcolor{blue}{==} , \textcolor{blue}{!=} , \textcolor{blue}{$<$} , \textcolor{blue}{$<$=} , \textcolor{blue}{$>$} , \textcolor{blue}{$>$=}

\begin{lstlisting}[style=frame, style= linenumbers, language=C]
class Date {
    int year, month, day; // private 

    bool operator<(Date const & rhs) const {
        return year < rhs.year ||
            (year == rhs.year && (month < rhs. month ||
                (month == rhs.month && day == rhs.day)));
    }
};

// With std::tie
#include <tuple>
bool operator<(Date const & rhs) const {
        return std::tie(year, month, day) < std::tie(rhs.year, rhs.month, rhs.day);
}

// Sending Date to std::ostream
class Date {
    int year, month, day;
public:
    std::ostream & print(std::ostream & os) const {
        os << year << "/" << month << "/" << day;
        return os;
    }
};
inline std::ostream & operator<<(std::ostream & os ,Date const & date) {
    return date.print(os);
}

// Reading Date from std::istream
class Date {
    int year, month, day;
public:
    std::istream & read(std::istream & is) {
        // Logic for reading values and verifying correctness
        return is;
    }
};
inline std::istream & operator>>(std::istream & is, Date & date) {
    return date.read(is);
}

/* Keyword friend, um Kapselung zwischen private/public zu brechen. --> Dann ist es möglich die read/print Operatoren ohne inline Hilfsfunktionen zu schreiben. */
// Header File:
class Date {
    int year, month, day;
public:
    friend std::istream & operator>>(std::istream & is, Date & date);
    friend std::ostream & operator<<(std::ostream & os, Date const & date);
};
// .cpp File:
std::istream & operator<<(std::istream & is, Date & date) {
    // read logic
    return is;
}
std::ostream & operator<<(std::ostream & os, Date const & date) {
    // print logic
    return os;
}
\end{lstlisting}


%! Author = joels
%! Date = 24/12/2020

\section{Namespaces and Enums}
% You can to group and structure names into namespaces
% You can resolve function calls using Argument Dependent Lookup
% You can create enumerations as simple types with few values
% You know how to implement your own arithmetic type and are aware of possible amibiguities

\subsection{Namespaces}
\begin{itemize}
    \item Namespaces are scopes for grouping and preventing name clashes
    \item Global namespace has the :: prefix
    \item Nesting of namespaces is possible
    \item Nesting of scopes allows hiding names
    \item Anonymous namespaces (without a name) are only accessible in the current file
    \item Keyword \textcolor{blue}{using} to use a defined namespace in your file.\\
    E.g: after typing \textbf{using std::string;} you can write: \textbf{string s$\{$\dq my string\dq$\}$;}
\end{itemize}

\subsubsection{Argument Dependent Lookup}
When the compiler encounters an unqualified function or operator call with an argument of a user defined type it looks into the namespace in which that type is defined to resolve the function/operator.
\begin{lstlisting}[style=frame, style= linenumbers, language=C]
namespace one {                                  #include "adl.h"
    struct type_one{};
    void f(type_one) { /* ... */ } /* 1 */       int main() {
}                                                    one::type_one t1{};
                                                     f(t1); // Function 1
namespace two {                                      two::type_two t2{};
struct type_two{};                                   f(t2); // Function 2
    void f(type_two) { /* ... */ } /* 2 */           h(t1); // No Function found
    void g(one::type_one) { /* ... */ } /* 3 */      two::g(t1); // Function 3
    void h(one::type_one) { /* ... */ } /* 4 */      g(t1); // No Function found
}                                                    //(5 gefunden, aber arg passt nicht)
void g(two::type_two) { /* ... */ } /* 5 */          g(t2) // Function 5
\end{lstlisting}

\subsection{Enums}
\begin{itemize}
    \item Enumerations are useful to represent types with only a few values
    \item An enumeration creates a new type that can easily be converted to an integral type unscoped enumeration only)
    \item The individual values (enumerators) are specified in the type
    \item Unless specified explicitly, the values start with 0 and increase by 1
\end{itemize}

\begin{lstlisting}[style=frame, style= linenumbers, language=C]
// Unscoped enumeration, leaks into surrounding scope
enum DayOfWeek {
    Mon, Tue, Wed, Thu, Fri, Sat, Sun
};   0    1    2    3    4    5    6
// Implicit conversion:
int day = Sun;

// Scoped enumeration (class keyword), requires DayOfWeek::
enum class DayOfWeek {
    Mon, Tue, Wed, Thu, Fri, Sat, Sun
};   0    1    2    3    4    5    6
// No implicit conversion to int, requires static_cast:
int day = static_cast<int>(DayOfWeek::Sun);

// Conversion from int to enum always requires a static_cast:
DayOfWeek tuesday = static_cast<DayOfWeek>(1);
\end{lstlisting}
\vspace{0.3cm}
\textbf{$\rightarrow$ Beispiel im Anhang unter Woche06}

\subsection{Arithmetic Types}
\begin{itemize}
    \item Arithmetic types must be equality comparable
    \item Boost can be used to get != operator $\rightarrow$ boost::equality\_comparable
    \item It might be convenient to have the output operator
    \item Result must be in a specific range (Modulo)
\end{itemize}
\vspace{0.3cm}
\textbf{$\rightarrow$ Beispiel im Anhang unter Woche06}


%TODO: Ring Implementation in Anhang
%! Author = joels
%! Date = 02/02/2021

\subsection{Testat2: KWIC}

\subsubsection{word.h}
\begin{lstlisting}[style=frame, style= linenumbers, language=C]
#ifndef WORD_H_
#define WORD_H_

#include <algorithm>
#include <cctype>
#include <iosfwd>
#include <string>

namespace text {

class Word {
	std::string value{"word"};

public:
	Word() = default;
	explicit Word(std::string);
	std::istream & read(std::istream &is);
	std::ostream & print(std::ostream &os) const;
	bool operator<(Word const & other) const {

	return std::lexicographical_compare(value.begin(), value.end(), other.value.begin(), other.value.end(), [](char c1, char c2) {
		return tolower(c1) < tolower(c2);
		});
	}

};

inline std::istream & operator>>(std::istream & is, Word & word) {
    return word.read(is);
}
inline std::ostream& operator <<(std::ostream &os, Word const &word) {
	return word.print(os);
}

inline bool operator>(Word const & word1, Word const & word2){
	return word2 < word1;
}

inline bool operator>=(Word const & word1, Word const & word2){
	return !(word1 < word2);
}

inline bool operator<=(Word const & word1, Word const & word2){
	return !(word2 < word1);
}

inline bool operator==(Word const & word1, Word const & word2){
	return !(word1 < word2) && !(word2 < word1);
}

inline bool operator!=(Word const & word1, Word const & word2){
	return !(word1 == word2);
}

}

#endif /* WORD_H_ */
\end{lstlisting}

\subsubsection{word.cpp}
\begin{lstlisting}[style=frame, style= linenumbers, language=C]
#include "word.h"
#include <istream>
#include <stdexcept>
#include <string>
#include <algorithm>
#include <cctype>
#include <exception>

namespace text {

Word::Word(std::string value)
: value(value) {
	if (value.empty()) {
		throw std::invalid_argument{"Invalid Argument"};
	}
	std::for_each(value.begin(), value.end(), [](char const c) {
		if (!std::isalpha(c)) {
			throw std::invalid_argument{"Invalid Argument"};
		}
	});
}

std::istream & Word::read(std::istream & is) {
    std::string value{};
    char c{};

    while(!std::isalpha(is.peek()) && is.good()) {
    	is.ignore(1);
    	if (is.eof()) {
    		is.setstate(std::ios::failbit);
    		break;
    	}
    }
    while(std::isalpha(is.peek())) {
    	is >> c;
    	value += c;
    }

    try {
        Word input{value};
        (*this) = input;
    } catch(std::exception const & e) {
        is.setstate(std::ios::failbit);
    }
    return is;
}

std::ostream & Word::print(std::ostream & on) const {
    return on << value;
}

}
\end{lstlisting}

\subsubsection{kwic.h}
\begin{lstlisting}[style=frame, style= linenumbers, language=C]
#ifndef SRC_KWIC_H_
#define SRC_KWIC_H_

#include "word.h" //Die Includes könnten reduziert werden, wenn die Hilfsfunktionen hier nicht deklariert wären
#include <iosfwd>
#include <vector>

namespace text {
using wordVector = std::vector<Word>;

void kwic(std::istream &in, std::ostream &out); //Gut

//Die Hilfsfunktionen sind ein Implementationsdetail und sollten nicht im Header deklariert werden (sofern nicht getestet)
std::vector<wordVector> createCombinations(std::vector<wordVector>);

void print(wordVector const line, std::ostream &out);

}


#endif /* SRC_KWIC_H_ */
\end{lstlisting}
\subsubsection{kwic.cpp}
\begin{lstlisting}[style=frame, style= linenumbers, language=C]
#include "kwic.h" //Gut, eigener Include zuerst
#include "word.h"
#include <istream> //Gut
#include <algorithm>
#include <sstream>
#include <string>
#include <vector>

namespace text {
using wordVector = std::vector<Word>; //Ein besserer Name wäre line

void kwic(std::istream &in, std::ostream &out) {
	//Das Einlesen könnte auch in eine eigene Funktion ausgelagert werden
	std::string lineStr{};
	std::vector<wordVector>lines{};
	while(std::getline(in, lineStr)) { //Gut
		std::istringstream is{lineStr};
		Word word{};
		wordVector line{}; //Die Variable könnte direkt mit einem Iterator-Paar initialisiert werden
		while (is >> word) {
			line.push_back(word);
		}
		lines.push_back(line);
	}
	std::vector<wordVector> combinations = createCombinations(lines);
	std::sort(combinations.begin(), combinations.end());

	std::for_each(combinations.begin(), combinations.end(), [&out] (auto line) {
		print(line, out);
	});

}

std::vector<wordVector> createCombinations(std::vector<wordVector> inputLines) { //Rotations
	std::vector<wordVector> combinations{};
	std::for_each(inputLines.begin(), inputLines.end(), [&combinations] (wordVector line) {
		for (unsigned i = 0; i < line.size(); i++) { //Ok
			combinations.push_back(line);
			std::rotate(line.begin(), line.begin() + 1, line.end());
		}
	});
	return combinations;
}

void print(std::vector<Word> const line, std::ostream &out) {
	std::for_each(line.begin(), line.end(), [&out] (Word word){ //Hier könnte besser mit dem copy-Algorithmus gearbeitet werden.

		out << word << " ";
	});
	out << "\n";
}
\end{lstlisting}
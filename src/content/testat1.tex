%! Author = joels
%! Date = 02/02/2021

\subsection{Testat1: 7 SEGMENT}
\subsubsection{sevensegment.h}
\begin{lstlisting}[style=frame, style= linenumbers, language=C]
#ifndef SEVENSEGMENT_H_
#define SEVENSEGMENT_H_

#include <iosfwd>

void printLargeDigit(int i, std::ostream &out);
void printLargeNumber(int i, std::ostream &out);
void printLargeError(std::ostream &out);

#endif /* SEVENSEGMENT_H_ */
\end{lstlisting}

\subsubsection{sevensegment.cpp}
\begin{lstlisting}[style=frame, style= linenumbers, language=C]
#include "sevensegment.h"
#include <ostream>
#include <array>
#include <string>
#include <algorithm>
#include <iterator>

const std::array<std::array<std::string, 5>, 11> digits{{
	{{" - ", "| |", "   ", "| |", " - "}}, //0
	{{"   ", "  |", "   ", "  |", "   "}}, //1
	{{" - ", "  |", " - ", "|  ", " - "}}, //2
	{{" - ", "  |", " - ", "  |", " - "}}, //3
	{{"   ", "| |", " - ", "  |", "   "}}, //4
	{{" - ", "|  ", " - ", "  |", " - "}}, //5
	{{" - ", "|  ", " - ", "| |", " - "}}, //6
	{{" - ", "  |", "   ", "  |", "   "}}, //7
	{{" - ", "| |", " - ", "| |", " - "}}, //8
	{{" - ", "| |", " - ", "  |", " - "}}, //9
	{{"   ", "   ", " - ", "   ", "   "}}  //-
}};

const std::array<std::string, 5> error {{ //ALF ist wohl nicht so happy, da einiges an whitespace fehlt
	{" -"},
	{"|"},
	{" -  -  -  -  -"},
	{"|  |  |  | ||"},
	{" -        -"}
}};

//Gut
void printLargeDigit(const int i, std::ostream &out) {
	//Es sollte noch geprüft werden, ob i im Range von 0 bis 9 ist
	std::copy(digits.at(i).begin(), digits.at(i).end(), std::ostream_iterator<std::string>{out, "\n"});
}

//Gut gelöst
void printLargeNumber(int i, std::ostream &out) {
    std::array<std::string, 5> lines{};
    std::string s = std::to_string(i);
    for(auto digit : s) {
    	int number{};
    	if (digit == '-') {
    		number = 10;
    	}
    	else {
    		number = digit - '0';
    	}
        for(unsigned i = 0; i < lines.size(); i++) {
            lines.at(i) += digits.at(number).at(i);
        }
    }
    std::copy(lines.begin(), lines.end(), std::ostream_iterator<std::string>{out, "\n"});
}
//Sehr gut gelöst
void printLargeError(std::ostream &out) {
	std::copy(error.begin(), error.end(), std::ostream_iterator<std::string>{out, "\n"});
}
\end{lstlisting}

\subsubsection{calc.h}
\begin{lstlisting}[style=frame, style= linenumbers, language=C]
#ifndef SRC_CALC_H_
#define SRC_CALC_H_

#include <iosfwd> //Minimaler Include: Korrekt

int calc(int first, int second, char opr);
int calc(std::istream & in);

#endif /* SRC_CALC_H_ */
\end{lstlisting}

\subsubsection{calc.cpp}
\begin{lstlisting}[style=frame, style= linenumbers, language=C]
#include "calc.h" //Gut: Eigener Include zuerst
#include <exception> //Includes sind korrekt
#include <istream>
#include <stdexcept>

int calc(int const number1, int const number2, char const symbol) {
	switch (symbol) {
		case '+':
			return number1 + number2;
		case '-':
			return number1 - number2;
		case '*':
			return number1 * number2;
		case '/':
			if (number2 == 0) {
				//std::invalid_argument passt als Exception besser
				throw std::overflow_error("Divide By Zero Exception"); //Auch Exceptions mit {} initialisiseren.
			}
			return number1 / number2;
		case '%':
			if (number2 == 0) {
				throw std::overflow_error("Modulo By Zero Exception");
			}
			return number1 % number2;
		default:
			throw std::overflow_error("Invalid Operator Exception");
	}
}

int calc(std::istream & in) {
	int number1{};
	char op{};
	int number2{};
	//Gut mit dem Stream-State gearbeitet
	if (in >> number1 >> op >> number2) {
		return calc(number1, number2, op);
	}
	else {
		throw std::exception(); //Dies ist sehr generisch. Auch Exceptions mit {} initialisiseren.
	}
}
\end{lstlisting}

\subsubsection{pocketcalculator.h}
\begin{lstlisting}[style=frame, style= linenumbers, language=C]
#ifndef POCKETCALCULATOR_H_
#define POCKETCALCULATOR_H_

#include <iosfwd>
void pocketcalculator(std::istream &, std::ostream&);

#endif /* POCKETCALCULATOR_H_ */
\end{lstlisting}

\subsubsection{pocketcalculator.cpp}
\begin{lstlisting}[style=frame, style= linenumbers, language=C]
#include "pocketcalculator.h" //Gut: Eigener Header zuerst
#include "calc.h"
#include "sevensegment.h"
#include <ostream>
#include <sstream>
#include <stdexcept>
#include <string>

void pocketcalculator(std::istream &in, std::ostream &out) {
	std::string line{};
	while(std::getline(in, line)) { //Hier sollten die Exceptions gefangen und Error ausgegeben werden
		std::istringstream is{line};
		int number = calc(is);
		if (std::to_string(number).length() > 8) { //Die Magic Number sollte in eine Konstante ausgelagert werden
			throw std::overflow_error("Too Large Result Exception");
		}
		printLargeNumber(number, out);
	}
}
\end{lstlisting}

\subsubsection{main.cpp}
\begin{lstlisting}[style=frame, style= linenumbers, language=C]
    #include "calc.h" //Braucht es hier nicht
#include "pocketcalculator.h"
#include "sevensegment.h" //Braucht es hier nicht
#include <exception>
#include <iostream>

int main() {
	while (std::cin) { //Hier sollte es nur den Aufruf von pocketcalculator() geben. Alles andere ist nicht testbar
		try {
			pocketcalculator(std::cin, std::cout);
		}
		catch (std::exception const &e) {
			printLargeError(std::cout);
		}
	}
}
\end{lstlisting}
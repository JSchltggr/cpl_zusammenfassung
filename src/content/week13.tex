%! Author = joels
%! Date = 24/12/2020

\section{Initialization and Aggregates}
% You can recognize and name different kinds of initialization
% You can explain the constraints imposed on aggregate types
% You can implement aggregate classes

\subsection{Kinds of Initialization}
\subsubsection{Default Initialization}
\begin{itemize}
    \item Simplest form of initialization
    \begin{itemize}
        \item Simply dont provide an initializer
        \item Effect depends on the kind of entity we declare
        \item Does not work for references
    \end{itemize}
    \item Danger: when using default initialized entities
    \item Does not necessarily work with \textit{const}
    \item \textbf{Static values:}
    \begin{itemize}
        \item Zero initialized first
        \item then their type's default constructor is called
    \end{itemize}
    \item \textbf{Non-Static integral and floating point variables:}
    \begin{itemize}
        \item Unitialized
    \end{itemize}
    \item \textbf{Objects of class types:}
    \begin{itemize}
        \item Constructed using their default constructor
    \end{itemize}
    \item \textcolor{red}{Danger:} Reading an unitialized value incurs undefined behaviour
\end{itemize}
\begin{lstlisting}[style=frame, style= linenumbers, language=C]
int global_variable; // implicitly static = 0, kein default ctor

void di_function() {
    static long local_static;
    long local_variable; // unitialized
    std::string local_text; // default ctor
}

struct di_class {
    di_class() = default;
    char member_variable; // not in ctor init list, default init
};
\end{lstlisting}

\subsubsection{Value Initialization}
\begin{itemize}
    \item Initialization performed with empty () or \{\}
    \begin{itemize}
        \item \{\} is prefferable, works in more cases
    \end{itemize}
    \item Invoked the default constructor for class types
\end{itemize}

\begin{lstlisting}[style=frame, style= linenumbers, language=C]
void vi_function() {
    int number{}; // 0
    std::vector<int> data {}; // default ctor
    std::string actually_a_function(); // DANGER! may be a function declaration, use {}
}
\end{lstlisting}

\subsubsection{Direct Initialization}
\begin{itemize}
    \item Similar to Value Initialization
    \item Uses non-empty () or \{\}
    \begin{itemize}
        \item prefer \{\}
    \end{itemize}
    \item When using \{\} only applies if not a class type
    \item \textcolor{red}{Danger}: most vexing parse lurks with ()
\end{itemize}

\begin{lstlisting}[style=frame, style= linenumbers, language=C]
void diri_function() {
    int number{32};
    std::string text { "CPl" }; // passender ctor
    // zweideutig, direct initialization / function declaration, use {}
    word vexing (std::string());
}
\end{lstlisting}

\pagebreak

Most Vexing Parse:
\begin{itemize}
    \item Two interpretations:
    \begin{itemize}
        \item Initialization with a value-initialized string
        \item Declaration of a function returning a word and taking an unnamed pointer to a function returning a string
    \end{itemize}
    \item The compiler interprets the second option (function declaration)
    \item Therefore prefer \{\}
\end{itemize}

\subsubsection{Copy Initialization}
\begin{itemize}
    \item Initialization using =
    \item If the object has class type and the right hand side has the same type
    \begin{itemize}
        \item If right hand side is a temporary, the object is constructed in-place, not copied
        \item otherwise the copy ctor is invoked
    \end{itemize}
    \item Otherwise, a suitable conversion sequence is searched for
    \item Also applies to return statements and throw/catch
\end{itemize}

\begin{lstlisting}[style=frame, style= linenumbers, language=C]
std::string string_factory() { retzrb ""; }

void ci_function() {
    std::string in_place = string_factory();
    std::string copy = in_place;
    std::string converted = "CPl";
}
\end{lstlisting}

\subsubsection{List Initialization}
\begin{itemize}
    \item Uses non-empty \{\}
    \item Direct List Initialization
    \item Copy List Initialization
    \item Constructors are selected in two phases
    \begin{itemize}
        \item If there is a suitable ctor taking \textit{std::initializer\_list}, it is selected
        \item Otherwise, a suitable ctor is searched
    \end{itemize}
\end{itemize}

\begin{lstlisting}[style=frame, style= linenumbers, language=C]
// Direct List Initialization
std::string direct { "CPl" };
// Copy List Initialization
std::string copy = { "CPlA" };
\end{lstlisting}

\subsubsection{List Initialization Pitfall}
\begin{itemize}
    \item Since \textit{std::initializer\_list} ctor is preffered, you might run into trouble
\end{itemize}

\begin{lstlisting}[style=frame, style= linenumbers, language=C]
// vector(size_type count, const T& value, const Allocator& alloc = Allocator());

int ouch() {
    std::vector<int> data{ 10, 42 }; // list initialization, size() = 2
    std::vector<ing> data2( 10, 42 ); // other ctor, size() = 10

    return data[5]; // DANGER, undefined behaviour
}
\end{lstlisting}

\pagebreak

\subsection{Aggregates}
\begin{itemize}
    \item Simple class types
    \begin{itemize}
        \item Can have other types as public base classes
        \item Can have member variables and functions
        \item Must not have user-provided, inherited or explicit constructors
        \item Must not have protected or private direct members
    \end{itemize}
    \item Mostly used for \textit{simple} types
    \begin{itemize}
        \item No invariant that has to be established
        \item Example: Data Transfer Objects
    \end{itemize}
    \item Arrays are also Aggregates
\end{itemize}

\subsubsection{Aggregate Initialization}
\begin{itemize}
    \item Special case of List Initialization
    \item If more elements than members are given, the program is ill-formed
    \item Can also provide less initializers than there are bases and members
    \begin{itemize}
        \item If the uninitialized members habe a member initializer, it is used
        \item otherwise they are initialized from empty lists
    \end{itemize}
\end{itemize}

\subsubsection{Example}

\begin{lstlisting}[style=frame, style= linenumbers, language=C]
struct person {
    std::string name;
    int age{42};

    bool operator<(person const & other) const {
        return age < other.age;
    }

    void write(std::ostream & out) const {
        out << name << ": " << age << "\n";
    }
};

int main() {
    person rudolf{"Rudolf", 32};
    person mike{"Mike Schmid"}; // Age will be set to 42
    rudolf.write(std::cout);
}
\end{lstlisting}



















%! Author = joels
%! Date = 26/12/2020

\subsection{Übungen Woche 10}

\subsubsection{Adapting Standard Containers}
\begin{lstlisting}[style=frame, style= linenumbers, language=C]
template<typename T>
struct safeVector : std::vector<T> {
    using container = std::vector<T>;
    using container::container; // or using std::vector<T>::vector; --> Inherits constructors with using
    using size_type = typename container::size_type; // Type Alias
    using reference = typename container::reference; // Type Alias
    using const_reference = typename container::const_reference; // Type Alias

    reference operator[](size_type index) {
        return this->at(index); // this-> for member access
    }

    const_reference operator[](size_type index) const {
        return this->at(index);
    }
    // should also provide front/back with empty() check
};
\end{lstlisting}

\subsubsection{Template Sack}
\begin{lstlisting}[style=frame, style= linenumbers, language=C]
// sack.h
template<typename T> // Class template with one typename parameter
class Sack {
    using SackType = std::vector<T>; // Alias type: SackType
    using size_type = typename SackType::size_type; // Dependent name: size_type
    SackType theSack{};

public:
    bool empty() const {
        return theSack.empty{};
    }
    size_type size() const {
        return theSack.size();
    }
    void putInto(T const & item) {
        theSack.push_back(item);
    }
    T getOut(); // Member function forward declaration
};

template <typename T> // implementing member function outside of a class
inline T Sack<T>::getOut() {
    if (empty()) {
        throw std::logic_error{"Empty Sack"};
    }
    auto index = static_cast<size_type>(rand() % size()); // Pick random element
    T retval{theSack.at(index)};
    theSack.erase(theSack.begin() + index);
    return retval;
}
\end{lstlisting}

\subsubsection{Extending Sack$<$T$>$}
\begin{lstlisting}[style=frame, style= linenumbers, language=C]
template <typename T>
class Sack {
    //...
public:
    Sack() = default; // Retain default constructor

    // Add constructor template for iterators
    template <typename Iter>
    Sack(Iter begin, Iter end) : theSack(begin, end) {}

    // Add constructor for initializer_list
    Sack(std::initializer_list<T> values) : theSack(values) {}

    // Add constructor for Deduction Guide
    Sack(size_type n, T const & value) : theSack(n, value) {}

    // extracting a std::vector<T> from a Sack<T>
    template <typename Elt>
	explicit operator std::vector<Elt>() const {
		return std::vector<Elt>(theSack.begin(),theSack.end());
    }

};

// User Provided Deduction Guide for Sack<T> (after template definition)
template <typename Iter>
Sack(Iter begin, Iter end) -> Sack<typename std::iterator_traits<Iter>::value_type>;
\end{lstlisting}

\subsubsection{Template/Default/Non-Type Template Parameter}
\begin{lstlisting}[style=frame, style= linenumbers, language=C]
// A template can take templates as parameters
template<typename T, template<typename...> typename Container> // ... for unspecified amount of parameters
class Sack;

// with default argument:
template<typename T, template<typename...> typename Container = std::vector>
class Sack;

// Templates can have non type parameters
template <typename T, auto n>
auto average(std::array<T, n> const & values) {
    auto sumOfValues = accumulate(begin(values), end(values), 0);
    return sumOfValues / n;
}
\end{lstlisting}
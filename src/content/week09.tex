%! Author = joels
%! Date = 24/12/2020

\section{Function Templates}
% You can explain the difference between Java Generics and C++ Templates
% You can implement simple generic functions
% You can implement variadic generic functions
% You can describe the type concepts required by a function template

\subsection{Templates Syntax}
\begin{itemize}
    \item Keyword \textcolor{blue}{template} for declaring a template
    \item Template parameter list: Contains one or more template parameters
        \SubItem{A template parameter is a placeholder for a type, which can be used within the template as a type}
        \SubItem{A type template parameter is introduced with the typename (or class ) keyword}
\end{itemize}

\begin{lstlisting}[style=frame, style= linenumbers, language=C]
// min.h
template <typename T>
T min(T left, T right) {
    return left < right ? left : right;
}

// smaller.cpp
#include "min.h"
#include <iostream>
int main() {
    int first;
    int second;
    if (std::cin >> first >> second) {
        auto const smaller = min(first, second);
        std::cout<< "Smaller of "<< first<< " and "<< second<< " is: "<< smaller<< '\n';
    }
}
\end{lstlisting}

\subsubsection{Template Definition}
\begin{itemize}
    \item Templates are usually defined in a header file. They are implicitly \textcolor{blue}{inline}
    \item The definition in a source file is possible, but then it can only be used in that translation unit
    \item Type checking happens twice: When the template is defined, when template is instantiated (used)
\end{itemize}

\subsection{Variadic Templates}
\begin{itemize}
    \item For function templates with an arbitrary number of parameters
    \item Needs at least one pack parameter
    \item Pack Expansion: For each argument in that pack an instance of the pattern is created
    \item In an instance of the pattern the parameter pack name is replaced by an argument of the pack
    \item Needs a base case for the recursion (after the last parameter is done, it would call the function without a parameter, which is invalid) $\rightarrow$ Base case must be written before the template function.
\end{itemize}
\begin{lstlisting}[style=frame, style= linenumbers, language=C]
void printAll() { } // Base Case
template<typename First, typename...Types>
void printAll(First const & first, Types const &... rest) {
    std::cout << first;
    if (sizeof...(Types))
        std::cout << ", ";
    }
    printAll(rest...);
}
\end{lstlisting}

\subsection{Overloading}
Problem: Wenn Pointer als Argument übergeben werden, werden die Pointer-Adressen miteinander verglichen und nicht die Objekte selbst.\\
Lösung: Mehrere function templates mit dem Selben Namen können existieren. Spezifischere/Normale Funktionen werden den function templates aber bevorzugt.
\begin{lstlisting}[style=frame, style= linenumbers, language=C]
// other template function in min.h
template <typename T>
    T * min(T * left, T * right) {
        return *left < *right ? left : right;
}
\end{lstlisting}
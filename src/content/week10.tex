%! Author = joels
%! Date = 24/12/2020

\section{Class Templates}
% You know when and why to prefer standard algorithms over hand written loops
% You can name the most important algorithms in the STL
% You can explain certain pittfalls when using STL algorithms
% You can explain the signature of a standard algorithm
% You can write programs that correctly use standard algorithms

A class template provides a type with compile-time parameters.\\
$\rightarrow$ Data members can depend on template parameters.\\
$\rightarrow$ Function members are template functions with the class' template parameters

\subsubsection{Rules}
\begin{itemize}
    \item Define class templates completely in header files
    \item Member functions of class templates: Either in class template directly, or as inline function template in the same header file
    \item When using language elements depending directly or indirectly on a template parameter, you must
    specify typename when it is naming a type
    \item static member variables of a template class can be defined in header without violating ODR, even
    if included in several compilation units
\end{itemize}

\subsubsection{Type Aliases \& Dependent Names}
\begin{itemize}
    \item It is common for template definitions to define type aliases in order to ease their use
    \item Within the template definition you might use names that are directly or indirectly depending on the
    template parameter.
    \item Dependent Name: Compiler geht standardmässig davon aus, dass es sich im eine Variable, oder eine Funktion hanelt. Wenn es ein Typ ist (wie size\_type), muss das keyword \textcolor{blue}{typename} verwendet werden.
\end{itemize}

\subsubsection{Members Outside of Class Template}
Must have: template declaration, keyword inline, member signature, template ID as namescope.\\
$\rightarrow$ Example im Anhang

\subsubsection{Class Template that Inherit}
Rule: Always use this-$>$variable (or className::) to refer to inherited members in a template class.

\subsubsection{Template (Partial) Specialization}
partial: still using a template parameter, but provide (some) arguments\\
Explicit: providing all arguments with concrete types.\\
$\rightarrow$ Must declare non-specialized template first. Most specialized version that fits is used.
\begin{lstlisting}[style=frame, style= linenumbers, language=C]
// Partial Specialization               Explicit Specialization
template <typename T>                   template <>
struct Sack<T *>;                       struct Sack<char const *>;

// Prevent Creation of Sack<int *> --> Delete Destructor
template <typename T>
struct Sack<T *> {
    ~Sack() = delete;
}
\end{lstlisting}

\subsubsection{Extending Sack$<$T$>$}
\textbf{Implementation im Anhang Kapitel Übungen Woche 10}\\
\begin{itemize}
    \item Create a Sack$<$T$>$ using iterators to fill it
    \item Create a Sack$<$T$>$ of multiple default values
    \item Create a Sack$<$T$>$ from initializer list
    \item Obtain copy of contents in a std::vector (for iteration and inspection)
    \item Auto-deducting T for a Sack$<$T$>$ from initializer list $\rightarrow$ User Provided Deduction Guides
    \item Allow to vary the type of the container to be used
\end{itemize}